\documentclass[a4paper, 12pt]{article}

\usepackage[margin=2cm,top=2.7cm,bottom=2.7cm]{geometry}
\usepackage{german}
\usepackage[utf8]{inputenc}
\usepackage{lastpage}
% \usepackage{fancyheadings}
\usepackage{fancyhdr}
\usepackage{setspace}
\usepackage{comment}
\usepackage{graphicx}
\usepackage{calc}
\usepackage{ifthen}
\usepackage{boxedminipage}
\usepackage{amsfonts}
\usepackage{url}
\usepackage{listings}

\frenchspacing
\sloppy
\setlength{\parindent}{0pt}
\setlength{\parskip}{1ex}
\pagestyle{fancy}
 
\lhead{Sommersemester 2012} 
\chead{}
\rhead{Git-Workshop}

\lfoot{Hinnerk Haardt, Ulrich Hoffmann}
\cfoot[Seite \thepage\ von \pageref{LastPage}]{Seite \thepage\ von \pageref{LastPage}}
\rfoot{2012-07-18}
\renewcommand{\headrulewidth}{1pt}
\renewcommand{\footrulewidth}{1pt}

\newcounter{enumisave}
\setcounter{enumisave}{0}
\newenvironment{enumerate*}%
{\begin{enumerate}\setcounter{enumi}{\theenumisave}}%
{\setcounter{enumisave}{\theenumi}\end{enumerate}}

\newenvironment{remark}{\begin{em}}{\end{em}}

\begin{document}

\centerline{\huge\bf Git \& Github}\bigskip

Eine der Stärken von Git ist, dass es die Zusammenarbeit von 
EntwicklerInnen unterstützt: Sie können im großen Stil an Projekten arbeiten und geordnet Änderungen austauschen.

Git ist ein verteiltes Versionssystem, d.\,h.\ es gibt typischerweise nicht nur
ein lokales Repository bei Euch sondern weitere \emph{remote}-Repositories auf anderen Rechnern, mit denen Änderungen ausgetauscht werden können. Zwar wollen wir zunächst nur jeder in seinem lokalen Repository arbeiten, doch soll es für unsere späteren Experimente gleich mit einem remote-Repository verbunden sein.\footnote{Ein isoliertes lokales Repository kann man ganz in einem Verzeichnis eigener Wahl einfach mit \texttt{git init} erzeugen. Machen wir hier aber nicht.}

Github ist das \emph{Social Network} für Entwickler. Es verwaltet öffentlich zugängliche (und private) Repositories mit Git und bietet darüber hinaus zahlreiche Funktionen zur Projektverwaltung und Präsentation des Projekts im Netz.

\texttt{git status} ist Euer Freund. Wann immer Ihr in Git verloren seid, könnt ihr damit sehen, wie Git die Lage beurteilt. Seht Euch die Ausgabe an und versucht zu verstehen, was Git Euch damit sagen will.



\subsection*{Repositories anlegen: Clonen}\vspace{-1.5ex}
Die einfachste Art, eine Verbindung zwischen Repositories herzustellen, ist ein neues Repository als vollständige Kopie (inklusive der gesamten Historie) eines bestehenden Repositories anzufertigen. Das geschieht mit \texttt{git clone} \textsf{Repository-URL}.
Der Einfachheit halber nehmen wir ein schon bestehendes Repository bei Github:

\begin{enumerate*}
\item Bitte kopiert Euch das Repository \texttt{git-workshop} unter \texttt{github.com/fh-wedel}:\\ \texttt{git clone https://github.com/fh-wedel/git-workshop.git}

Damit gibt es bei Euch nun ein lokales Repository im Verzeichnis \texttt{git-workshop}, das Ihr nun bearbeiten könnt. Seht Euch an, welche Files Git zur Verwaltung erzeugt (Ihr müsst sie nicht im Detail verstehen, aber wo sie stehen ist schon interessant.)
\end{enumerate*}

\section{Jeder für sich: Arbeiten im lokalen Repository}

Unser hier zu entwickelndes Mini-Projekt soll ein Web-Server für 
den Git-Workshop selbst sein. Dafür brauchen wir auch ein paar Web-Seiten in
HTML. Die findet man in \texttt{src/html}.

\subsection*{Im aktuellen Branch arbeiten}\vspace{-1.5ex}
Ohne weitere Maßnahmen arbeitet Ihr auf dem \texttt{master}--Branch.
Es ist in Git üblich, mit eigenen Branches zu arbeiten und nicht direkt auf
\texttt{master}. Wir machen das jetzt trotzdem erstmal.

\begin{enumerate*}

\item Legt bitte ein Unterverzeichnis mit Eurem Namen in \texttt{src/html} an, erstellt dort bitte ein paar HTML-Files und bearbeitet sie so, wie Ihr wollt. Was sagt \texttt{git status}? Nehmt (einige) Files für Euren ersten Commit auf (das sog.\ \emph{stagen} mit \texttt{git add} \textsf{filename}, was sagt \texttt{git status}?) und macht den Commit (\texttt{git commit -m}\verb|"|\textsf{Commit-Meldung}\verb|"|).  Was sagt \texttt{git status}? 
\end{enumerate*}
Weitere nützliche Kommandos, um zu sehen, wie Eure Änderungen aussehen sind \texttt{git diff} (zeigt den Unterschied zwischen lokalen Änderungen und gestageten Files)
und \texttt{git show} (zeigt die Änderungen des letzen Commits).


\subsection*{Branchen}\vspace{-1.5ex}
Git macht es leicht, für einzelne Aufgaben Branches anzulegen, in denen die Arbeiten für diese Aufgabe zusammengefasst werden. Es ist üblicher Git-Stil, dass man für jede Teilaufgabe einen Branch anlegt, darin die notwendigen Änderungen macht und den Branch dann zurück in den Haupt-Branch merget. So kann man die Änderungen für einzelne Features voneinander trennen und auch gezielt und ausgewählt zur Verfügung stellen.
\begin{enumerate*}
\item Macht wieder einige Änderungen in Euren lokalen Files (noch kein Stagen oder Committen). Vielleicht fügt Ihr \texttt{style}-Angaben hinzu oder erzeugt sogar ein eigenes Stylesheet.

Nachdem Ihr Eure Änderungen gemacht habt, stellt Ihr fest, dass sie besser auf einem Branch aufgehoben sind (vielleicht ist die Aufgabe, einige Passagen in rot hervorzuheben).
Mit \texttt{git checkout -b} \textsf{Branch-Name} könnt Ihr nachträglich den
Branch erzeugen, in dem Ihr dann Eure Änderungen stagen und commiten könnt. 
Macht mal einen Branch und nennt Ihn einfach \texttt{texte-hervorheben}.

Was sagt eigentlich \texttt{git status}?
\item Arbeitet so auf Eurem neuen Branch und macht ein paar Commits.
\end{enumerate*}

Vielleicht stellt Ihr fest, dass die Arbeiten auf dem Branch in eine Sackgasse geführt haben und Ihr sie gar nicht weiterverwenden wollt. Dann kann man einfach den Branch Branch sein lassen und zurück auf \texttt{master} wechseln: \texttt{git checkout master} (natürlich kann man den Branch auch wieder löschen, muss man aber nicht). 

Aber meist, wollt Ihr die Änderungen im Branch weiterverwenden. Sie werden dafür ge\emph{merge}\/t. 

\subsection*{Mergen}\vspace{-1.5ex}
Das Mergen erfolgt bei Git immer in einen Branch hinein, d.\,h.\ man wechselt also erst in den Branch in den man mergen will und löst dann das Mergen aus.

\begin{enumerate*}

\item Übernehmt Eure Änderungen von Eurem Branch auf den \texttt{master}-Branch (\texttt{git checkout master; git merge --no-ff} \textsf{Branch-Name}\footnote{\texttt{--no-ff} bedeutet, dass bestimmt kein sog.\ \emph{Fast-Forward-Merge} gemacht wird, bei dem nicht wirklich gemerget wird, sondern nur Referenzen verschoben werden.}).

Wenn es keine konkurrierenden Änderungen auf dem \texttt{master}--Branch gibt,
dann sollte das Mergen ohne Probleme erfolgen und auch gleich commited werden.

Gibt es konkurrierende Änderungen, entstehen möglicherweise Konflikte. Git committed den Merge nicht, Ihr müsst die Konflikte sinnvollerweise beheben und
dann selbst (stagen und) committen.

\end{enumerate*}

\newpage
\section{Alle miteinander: Arbeiten mit entferntem Repository}
Git-Repositories können (und sind oft) mit anderen Repositories (sog.\ \emph{remotes}) verbunden. Zwischen diesen Repositories können Commits ausgetauscht
werden. 

Wir hatten unser Repository ja durch \texttt{git clone} erzeugt und dabei wird gleich eine Verbindung zum Ursprungs-Repository unter dem Namen \texttt{origin} hergestellt. Das kann man mit \texttt{git remote -v} sehen.


\subsection*{Pull und Push}\vspace{-1.5ex}
Mit \texttt{git pull} werden Commits aus einem Remote-Repository in das lokale
Repository übernommen. Mit \texttt{git push} werden Commits aus dem lokalen Repository in ein Remote-Repository übertragen.

\begin{enumerate*}
\item Macht bitte wie zuvor Änderungen in Eurem lokalen Repository (Eigener Branch, Änderungen, commit, Änderungen, commit, mergen) und pusht dann Eure Änderungen in das Github-\texttt{git-workshop}-Repository. Werden Eure lokalen Branches und die Commits darauf auch mitübertragen?

\item Bitte passt das File \texttt{src/html/index.html} an und tragt Euch in die Teilnehmerliste (gerne mit Pseudonym) ein. Stellt auch diese Eure Änderung bereit.

\end{enumerate*}


\vfill

\section*{Git Revisions}

Wie adressiere ich eine Revision?

\begin{itemize}
\item Mit dem SHA-Wert, der im Log anzeigt wird. Dieser kann von hinten abgeschnitten werden, so lange der Bezeichner im Repository eindeutig ist.
\item Mit der Ausgabe von \texttt{git describe}.
\item Mit einer symbolischen Referenz, die in im Repository definiert ist, z.B. \texttt{master}, \texttt{HEAD}, \texttt{FETCH\_HEAD}, \texttt{ORIG\_HEAD}
\item Mit einer Kombination aus Referenz und Datum: \texttt{master@\{yesterday\}}, \texttt{HEAD@\{5 minutes ago\}}. Diese Adressen liefern den Zustand zu dem spezifizierten Zeitpunkt, um Änderungen innerhalb eines Zeitraumes anzuzeigen eignen sich \texttt{--since} und \texttt{--until}.
\item Vorläufer einer Referenz: \texttt{HEAD~10}
\end{itemize}

\vfill
\section*{Nützliche Links}

Git Quick reference\\
\url{http://jonas.nitro.dk/git/quick-reference.html}


\section{Weitere Fähigkeiten von Git}

\begin{description}
  \item[bisect] \hfill \\
  Suche nach einem fehlerhaftem in einer Reihe von Commits. Mittels \texttt{git bisect start} wird bisect gestartet, mittels \texttt{git bisect bad} und \texttt{git bisect good} werden defekte und heile Fassungen markiert. Bisect teilt die Commits innerhalb der Markierungen auf und checkt eine Revision in der Mitte aus, die getestet und dann entsprechend markiert werden muß. Zum Schluß bleibt die defekte Revision übrig.
    
  Die Hilfe (\texttt{git help bisect}) bietet detaillierte Beschreibungen zu einer Reihe weiterer Befehle, z.B. zum Zurücksetzen des Zustandes, Betrachten der Historie, Überspringen von Commits etc.
  \item[bundle] \hfill \\
  Archiv-Datei aus einem Git-Repository erstellen. Komprimierte Bundle-Dateien stellen die kleinste Repräsentation eines Repositories dar. Daher eignen sie sich hervorragend für den Versand per E-Mail oder Transport auf einem USB-Stick. 
  
  Mit \texttt{git bundle create repo.bundle master} werden alle zum Branch \texttt{master} gehörenden commits in die Datei \texttt{repo.bundle} geschrieben. Es ist auch möglich, kleinere bundles zu erzeugen, z.B. mit \texttt{--since=10.days master} oder \texttt{v2.0..master}.
  
  Mit \texttt{clone}, \texttt{fetch} und \texttt{pull} werden die Dateien eingelesen, z.B. \texttt{git clone repo.bundle newrepo}. Weitere Befehle wie \texttt{verify} und \texttt{unbundle} werden in der Hilfe beschrieben.
  \item[cherry-pick] \hfill \\
  einige Änderungen übernehmen
  
  \item[grep] \hfill \\
  in Dateien und Archiv suchen
  \item[stash] \hfill \\
  Änderungen temporär speichern
  \item[svn] \hfill \\
  Interface zu Subversion

  \item[submodule] \hfill \\
	Repositories schachteln
  \item[notes] \hfill \\
  Kommentare an commit hängen
  \item[archive] \hfill \\
  extrahieren der Nutzdaten

  
  
  
\end{description}


\end{document}




